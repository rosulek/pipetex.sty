\documentclass{article}

\usepackage{pipetex}
\pipetexcommand{perl}

\begin{document}

  The contents of the \texttt{pipetex} environment will be
  passed as STDIN to whatever command you specified in
  \texttt{\textbackslash pipetexcommand}.
  The STDOUT output of the command will be rendered as \LaTeX
  at this point in the document:

  \begin{center}
    \begin{pipetex}
      printf '\framebox{%s}', $_ for 1 .. 20;
    \end{pipetex}
  \end{center}
  
  In these examples the contents of the environment are 
  sent to \texttt{perl}.  Here is another example:
 
  \begin{center}
    \begin{pipetex}
      my @fib = (0,1);
      for my $i (2 .. 18) {
        $fib[$i] = $fib[$i-1] + $fib[$i-2];
      }
      print join ", ", @fib;
    \end{pipetex}
  \end{center}
  
  You can also override the default command by 
  passing an optional argument to the environment,
  like this:
  
  \begin{center}
    \begin{pipetex}[sort]
      every
      good
      boy
      deserves
      fudge
    \end{pipetex}
  \end{center}

  When the command gives an error (as indicated by
  its return code), the contents of STDERR are rendered
  instead:

  \begin{center}
    \begin{pipetex}
        $this = 'is';
        nonsensical "Perl"
    \end{pipetex}
  \end{center}

  
\end{document}
